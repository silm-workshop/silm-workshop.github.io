% License:
% CC BY-NC-SA 3.0 (http://creativecommons.org/licenses/by-nc-sa/3.0/)
%
%%%%%%%%%%%%%%%%%%%%%%%%%%%%%%%%%%%%%%%%%

\documentclass[a4paper,11pt]{article} 
\usepackage[margin=2cm]{geometry}
\usepackage[T1]{fontenc} 
\usepackage[utf8]{inputenc}
\usepackage{fourier} 
\usepackage[english]{babel} 
\usepackage{amsmath,amsfonts,amsthm}
\usepackage{paralist}
\usepackage{caption}
\usepackage{subcaption}
\usepackage{graphicx}
\usepackage{float}
\usepackage{blindtext} 
\usepackage{tabularx}
\usepackage{tabto}
\usepackage{boldline}

\usepackage{verbatim}
\usepackage{etoolbox}
\usepackage{todonotes}
\usepackage{nth}

\PassOptionsToPackage{hyphens}{url}
\usepackage[]{hyperref} 

\usepackage{fancyhdr} 
\pagestyle{fancyplain}
\fancyhead{} 
\fancyfoot[L]{} 
\fancyfoot[C]{}
\fancyfoot[R]{\thepage} 
\renewcommand{\headrulewidth}{0pt}
\renewcommand{\footrulewidth}{0pt} 
\setlength{\headheight}{13.6pt} 

\setlength\parskip{1pt}

\usepackage{authblk}
\newcommand{\horrule}[1]{\rule{\linewidth}{#1}} 

\newcommand\etal{\emph{et al.}\xspace}
\newcommand\eg{\emph{e.g.}\xspace}
\newcommand\ie{\emph{i.e.}\xspace}



% ---------------------------------------------------------------------------
\newcommand{\SILMNum}{\nth{5}}
\newcommand{\SILMYear}{2023}
\newcommand{\SILMDate}{Monday, the \nth{3} of July 2023}

\title{SILM Workshop on Security of Software/Hardware Interfaces}

\date{}

\author{Guillaume Hiet}
\affil{CIDRE team, CentraleSupélec/Inria, IRISA, France}

\author{Jan Tobias M\"uhlberg}
\affil{imec-DistriNet, KU Leuven, Belgium}
% ---------------------------------------------------------------------------



% ---------------------------------------------------------------------------
% ---------------------------------------------------------------------------
% ---------------------------------------------------------------------------
\begin{document}
\maketitle

\begin{center}
 \textbf{Contacts}:  \url{guillaume.hiet@centralesupelec.fr},
\url{jantobias.muehlberg@cs.kuleuven.be}
\end{center}

\noindent This is a proposal for a \emph{full-day} EuroS\&P \SILMYear{}
workshop, preferably to be held before the main conference, on
\emph{\SILMDate}.

% ---------------------------------------------------------------------------
% ---------------------------------------------------------------------------
\section{Aim, Scope and Context}
%
% ---------------------------------------------------------------------------
\subsection{Context and Past Events}
%
The security of software and hardware components used to be considered as
different problems and have been studied by distinct scientific
communities. However, it is increasingly important to combine both software
and hardware aspects of computer science to deal with new software attacks
that can be launched remotely and do not require any physical access to the
device. This emphasises the need to study both the attack and the defence
aspects of the security of software/hardware interfaces.

This poses several challenges. First, hardware platforms tend to be more
and more complex and the behaviour specifications of the different hardware
components are not always publicly available. Such specifications can also
be incomplete, incorrect, or imprecise. Thus, different approaches have to
be proposed by researchers to recover the expected behaviour or the state of
hardware components.

Another important challenge is to assess the security level of these
hardware platforms against software attacks. The study of existing and new
types of vulnerabilities in hardware platforms is crucial. Analysing
software attacks that can use such vulnerabilities is also important to
evaluate their real-world feasibility and impact.

As a third pillar, the study of defence mechanisms is also fundamental to
building secure systems. Modern platforms tend to embed more and more
security mechanisms in hardware components. This raises interesting
questions such as: What type of security mechanisms could be implemented in
hardware platforms at an acceptable cost? What are precisely the security
properties guaranteed by such mechanisms? Software countermeasures can also
be used in systems and applications executed on the hardware platform to
protect them against software attacks exploiting hardware vulnerabilities.

Guillaume Hiet and Jan Tobias M\"uhlberg are co-organising the SILM
workshop since 2021.
Guillaume Hiet organised a thematic
research semester on these topics\footnote{\url{https://silm.inria.fr/}},
called SILM hereafter. This semester was funded by the
DGA\footnote{\url{https://www.defense.gouv.fr/dga}}, the Government Defence
procurement and technology agency. It was  operated by
Inria\footnote{\url{https://www.inria.fr/en/}}, the French national
research institute for the digital sciences, on behalf of the partners of
the PEC (\textit{Pôle d'Excellence
Cyber})\footnote{\url{https://www.pole-excellence-cyber.org/presentation-du-pole/pole-dexcellence-cyber-at-a-glance/}}
General Partnership Agreement. The PEC is a French cluster of governmental,
industrial and academic partners on cyber-security.

The SILM thematic semester was dedicated to the security of
software/hardware interfaces and more particularly on the three
following axes:

\begin{enumerate}
%
    \item Analysing the behaviour and the state of hardware components
using, e.g. trace mechanisms, fuzzing, reverse-engineering techniques, or
side channel analyses;
%
    \item Studying the hardware vulnerabilities and the software attacks
that can exploit them: e.g. side-channels, fault injections, or
exploitation of unspecified behaviour;
%
    \item Detecting and preventing software attacks using dedicated
hardware components. Proposing software countermeasures to protect from
hardware vulnerabilities.
%
\end{enumerate}

In the context of this semester, we have already organised different events
on these topics:
%
\begin{itemize}
%
    \item A SILM Summer School (July 8-12, 2019 ):
\url{https://silm-school.inria.fr/}
%
    \item A \nth{1} edition of the SILM Workshop (November, 2019):
\url{https://silm-workshop.inria.fr/}
%
    \item A \nth{2} edition of the SILM Workshop (September 2020, with
EuroS\&P): \\ \url{https://silm-workshop-2020.inria.fr/}
%
    \item The \nth{3} and \nth{4} edition of the SILM Workshop (2021 and
2022, both with EuroS\&P, together with Jan Tobias M\"uhlberg): \\
\url{https://silm-workshop-2021.inria.fr/};
\url{https://silm-workshop.github.io/}
%
    \item A regular seminar: \url{https://silm-seminar.gitlabpages.inria.fr/}
%
\end{itemize}

The first edition of the SILM Workshop was organised during a more general
event, the European Cyber
Week\footnote{\url{https://www.european-cyber-week.eu/}}.  The \nth{2},
\nth{3} and \nth{4} edition were co-located with IEEE Euro S\&P as of 2020. 
% 
The summer school, the seminar and the first edition of the SILM workshop
were organised in Rennes, France. Attending those events was free of
charge. The \nth{2} and \nth{3} edition of the SILM workshop were organised
as online-events with IEEE Euro S\&P. Finally, the \nth{4} edition took
place as an in-person event again, with an ad-hoc hybrid component.

The programs of the events are available on their respective websites. We
also published the slides and the video recordings of the presentations on
our websites, if the speakers allowed us to do that. All the presentations
have been given by invited speakers from academia, governmental agencies
and industry. The number of participants was the following:
%
\begin{itemize}
%
    \item SILM School: 45 to 70 participants depending on the sessions (45
participants were registered for the whole event, including the labs),
including 13 invited speakers;
%
    \item \nth{1} edition of the SILM Workshop: 76 participants, incl.
12 invited speakers.
%
    \item \nth{2} edition of the SILM Workshop: 17 participants, incl.
2 invited speakers.
%
    \item \nth{3} edition of the SILM Workshop: $\approx$ 25 participants,
incl. 3 invited speakers and 2 extra panellists.
%
    \item \nth{4} edition of the SILM Workshop: $\approx$ 30 participants,
incl. 4 invited speakers.
%
\end{itemize}

We would like to organise a \SILMNum{} edition of the SILM workshop, again
co-located with EuroS\&P, soliciting short papers and full papers for novel
and unpublished with a focus on ongoing research with a potential to spark
discussions and new collaborations. SILM will also be an opportunity for
young researchers to present their ongoing work.

% ---------------------------------------------------------------------------
\subsection{Aim and Scope of the Proposed Workshop}
%
The purpose of the SILM workshop is to share experience, tools and
methodology to handle security in software/hardware interfaces. On one
hand, we need to better assess the security guarantees provided by existing
hardware architectures against software attacks, especially attacks against
micro-architecture. This can be achieved by identifying new vulnerabilities
using reverse engineering, fuzzing or other attack approaches. On the other
hand, we also need to propose new architectures offering a better
resilience against software attacks. These architectures should rely on
hardware-based security mechanisms to protect the software stack. One of
the challenges is to formally specify and verify the security guarantees
offered by such architectures.

The goal of this \SILMNum{} edition of the SILM workshop is to provide a forum
for  researchers and practitioners from academia, industry and government
that work on the security of software/hardware interfaces.

\paragraph{Special Theme \SILMYear{}: SILM for Telecommunications.}
%
As a special theme for SILM \SILMYear{}, we solicit submissions that
present novel research that develops or relies on hardware/software
interfaces to improve security and dependability in the telecommunications
sector. In particular, work that aims at 5G, edge and cloud scenarios where
strict requirements on bandwidth, latency, and availability need to be met,
are of special interest for SILM this year. Submissions must be within the
general scope of SILM \SILMYear{} as stated above.

\paragraph{Expected Number of Participants.}
%
According to the audience of the previous workshops and events organised
during the SILM semester, we expect 20 to 40 participants, in particular if
we would be able to hold a physical meeting in Delft.

\paragraph{Expected Number of Submissions.}
%
For the last two editions of the SILM workshop, we received 7 submissions
and we accepted one full paper (10 pages) and four short paper (6 pages) in
2021, and four short papers in 2022. We complemented the programme two to
four invited speakers and a panel discussion.


% ---------------------------------------------------------------------------
% ---------------------------------------------------------------------------
\section{Format}
%
We would like to organise a one day event mixing presentation from invited
speakers, regular research papers and short
papers/work-in-progress/industry experience reports. The precise format
will depend on the number of submissions but we target the following
format:
%
\begin{itemize}
    \item 3 invited speakers (45 minutes)
    \item 2 full research papers (25 minutes)
    \item 4 short presentations (15 minutes)
    \item 1 panel debate or discussion round (50 minutes)
\end{itemize}

Depending on the submissions, we could accept more research papers and
fewer short presentations. Expenses not covered by the support of Euro S\&P
will be covered by the budget of the SILM thematic semester or KU Leuven.

The proposed agenda is the following one:
%
\begin{itemize}
    \item 08:30 - 08:45 : Registration
    \item 08:45 - 09:00 : Opening remarks
    \item 09:00 - 10:00 : Keynote
    \item 10:00 - 10:30 : Coffee break
    \item 10:30 - 12:30 : Presentations
    \item 12:30 - 14:00 : Lunch
    \item 14:00 - 15:00 : Keynote
    \item 15:00 - 16:00 : Panel debate / Discussion / Short talks
    \item 16:00 - 16:30 : Coffee break
    \item 16:30 - 17:30 : Keynote
    \item 17:30 - 18:00 : Closing remarks
\end{itemize}
%
Of course, we will adapt this agenda depending on the local organisation constraints.


% ---------------------------------------------------------------------------
% ---------------------------------------------------------------------------
\section{Reviewing \& Publication}
%
We would like to propose two categories of submissions:
%
\begin{itemize}
%
    \item Regular papers describing fully developed work and complete
results (10 pages, references included, IEEE format);
%
    \item Short papers, position papers, industry experience reports,
work-in-progress submissions (6 pages, references included, IEEE format)
%
\end{itemize}

All papers should be in English and describe original work that has not
been previously published or submitted elsewhere. The submission category
should be clearly indicated. All submissions will be fully reviewed by
members of the Program Committee.

We expect between 10 and 20 submissions for SILM \SILMYear{}. Each
submission will be reviewed by three different members of the PC.  If
possible, papers will appear in IEEE Xplore in a companion volume to the
regular Euro S\&P proceedings.

In the past years we offered authors the option to present their papers but
not have them published in the official Euro S\&P proceedings, so as to
encourage authors to present and discuss ongoing work and leaving them the
option to publish their research once it is completed. We will also follow
this policy in \SILMYear{}.


% ---------------------------------------------------------------------------
% ---------------------------------------------------------------------------
\section{Organisation}
%
% ---------------------------------------------------------------------------
\subsection{Organisation Committee}
%
The Organisation Committee will be chaired by Guillaume Hiet and Jan Tobias
M\"uhlberg. Merve G\"ulmez will support the committee as Publicity Chair.

\textbf{Guillaume Hiet\footnote{\url{http://guillaume.hiet.fr/}}} works as
Professor at CentraleSupélec and a member of the IRISA/Inria CIDRE research
team in Rennes Brittany, France. Prior to that he was a computer security
expert at AMOSSYS SAS, a French security consulting company, from 2008 to
2010. He holds a PhD from Rennes 1 University, a Diplôme d'Ingénieur (MEng)
and a Master Recherche (MSc) from Supélec as well as a Diplôme d'Ingénieur
from ENSAM. He teaches cyber-security at CentraleSupélec and in other
institutions. His research interests are in network and computer security,
with a focus on security monitoring. He is particularly interested in the
security of low-level components such as OS, firmware and hardware. He was
involved in the PC of RAID 2011 and RAID 2012 security conference. He is
the chair holder of the SILM thematic semester.

\textbf{Jan Tobias
M\"uhlberg\footnote{\url{https://distrinet.cs.kuleuven.be/people/JanTobiasMuhlberg}}}
works as Professor for Embedded Systems Security at Universit\'e Libre de
Bruxelles (BE) and as Research Manager for Embedded Systems Security at
imec-DistriNet at the Department of Computer Science of KU Leuven (BE). He
is active in the fields of embedded systems security, hardware/software
co-design for security, verification and validation of software systems.
Tobias is particularly interested in security architectures for
safety-critical embedded systems and for the Internet of Things, and in the
concept of sustainability in the information and communications technology,
specifically in the context of security and privacy. Before joining
DistriNet he worked as a researcher at the University of Bamberg (DE),
obtained his Ph.D. from the University of York (UK) and worked as a
researcher at the University of Applied Sciences in Brandenburg (DE), where
he also obtained his Masters degree. Tobias has been a co-chair of SILM
since 2021, is a PC member of Euro S\&P 2023, is a co-organiser of the SICT
Summer School (since 2020), the QA\&Test conferences (since 2018), and
several seminar series at KU Leuven.

\textbf{Merve
G\"ulmez\footnote{\url{https://gulmezmerve.github.io/}}}
is a dependability researcher at Ericsson Research (SE) and KU Leuven (BE).
Merve is currently doing her PhD on low-level security and availability
mechanisms for cloud- and 5G telecommunications infrastructures.


% ---------------------------------------------------------------------------
\subsection{Draft Program Committee}
%
We are working on the composition of the Program Committee. We currently
strive to increase the geographic distribution and diversity of this
workshop's PC and the invited speakers. As of now we have contacted the
following distinguished researchers in the field, most of whom have
accepted to be part of the Program Committee of this workshop:

% Invite:
% Alasdair Armstrong, Cambridge
% David Goltzsche (Braunschweig) -> https://www.doc.ic.ac.uk/~vsartako/
% someone from the Robert Watson/Simon Moore group (Cambridge)
% Shweta Shinde or Srdjan Capkun (ETH)
% Cl\'ementine Maurice

\begin{itemize}
    \item Pascal Cotret, ENSTA Bretagne
    \item Chris Dalton, HP Labs
    \item{Lesly-Ann Daniel, KU Leuven}
    \item{Merve G\"ulmez, Ericsson Research/KU Leuven (Publicity)}
    \item Karine Heydemann, LIP6
    \item Guillaume Hiet, CentraleSupélec/Inria (Co-Chair)
    \item Vianney Lap\^otre, Univ. South Brittany
    \item{Cl\'ementine Maurice, Inria}
    \item Jan Tobias M\"uhlberg, KU Leuven (Co-Chair)
    \item Cristofaro Mune, Raelize B.V.
    \item Simon Rokicki, ENS Rennes
    \item Volker Stolz, HVL
    \item{Marcus V\"olp, Uni Luxembourgh}
    \item Pierre Wilke, CentraleSup\'elec/Inria
    \item Yuval Yarom, University of Adelaide and Data61
%    \item Guillaume Bouffard, ANSSI
%    \item Guy Gogniat, Univ. South Brittany
%    \item Damien Courouss\'e, CEA
%    \item Lucas Davi, University of Duisburg-Essen
%    \item Jean-Louis Lanet, Inria
%    \item Yves-Alexis Perez, ANSSI
%    \item Kaveh Razavi,  ETH Zürich 
%    \item Jan Reineke, Saarland University
%    \item Erven Rohou, Inria
%    \item Andr\'e Seznec, Inria
%    \item Arnaud Tisserand, CNRS
%    \item Fr\'ed\'eric Tronel, CentraleSup\'elec/Inria
%    \item Yossi Oren, Ben-Gurion University
\end{itemize}


% ---------------------------------------------------------------------------
% ---------------------------------------------------------------------------
\section{Draft CFP}
%
\verbatiminput{cfp.txt}


\end{document}
% ---------------------------------------------------------------------------
% ---------------------------------------------------------------------------
% ---------------------------------------------------------------------------

