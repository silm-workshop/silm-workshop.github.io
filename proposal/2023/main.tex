% License:
% CC BY-NC-SA 3.0 (http://creativecommons.org/licenses/by-nc-sa/3.0/)
%
%%%%%%%%%%%%%%%%%%%%%%%%%%%%%%%%%%%%%%%%%

%----------------------------------------------------------------------------------------
%	PACKAGES AND OTHER DOCUMENT CONFIGURATIONS
%----------------------------------------------------------------------------------------

\documentclass[a4paper,11pt]{article} % A4 paper and 11pt font size
\usepackage[margin=2cm]{geometry}
\usepackage[T1]{fontenc} % Use 8-bit encoding that has 256 glyphs
\usepackage[utf8]{inputenc}
\usepackage{fourier} % Use the Adobe Utopia font for the document - comment this line to return to the LaTeX default
\usepackage[english]{babel} % English language/hyphenation
\usepackage{amsmath,amsfonts,amsthm} % Math packages
\usepackage{lipsum} % Used for inserting dummy 'Lorem ipsum' text into the template
\usepackage{paralist}
\usepackage{caption}
\usepackage{subcaption}
\usepackage{graphicx}
\usepackage{float}
\usepackage{blindtext} %for enumerations
\usepackage{tabularx}
\usepackage{tabto}
\usepackage{boldline}

\usepackage{verbatim}
\usepackage{etoolbox}
\usepackage{todonotes}
\usepackage{nth}

\PassOptionsToPackage{hyphens}{url}
\usepackage[]{hyperref}  %link color

%table layout to the right
%\usepackage[labelfont=bf]{caption}
%\captionsetup[table]{labelsep=space,justification=raggedright,singlelinecheck=off}
%\captionsetup[figure]{labelsep=quad}

%\usepackage{sectsty} % Allows customizing section commands
%\allsectionsfont{\centering \normalfont\scshape} % Make all sections centered, the default font and small caps


\usepackage{fancyhdr} % Custom headers and footers
\pagestyle{fancyplain} % Makes all pages in the document conform to the custom headers and footers
\fancyhead{} % No page header - if you want one, create it in the same way as the footers below
\fancyfoot[L]{} % Empty left footer
\fancyfoot[C]{} % Empty center footer
\fancyfoot[R]{\thepage} % Page numbering for right footer
\renewcommand{\headrulewidth}{0pt} % Remove header underlines
\renewcommand{\footrulewidth}{0pt} % Remove footer underlines
\setlength{\headheight}{13.6pt} % Customize the height of the header

\numberwithin{equation}{section} % Number equations within sections (i.e. 1.1, 1.2, 2.1, 2.2 instead of 1, 2, 3, 4)
\numberwithin{figure}{section} % Number figures within sections (i.e. 1.1, 1.2, 2.1, 2.2 instead of 1, 2, 3, 4)
\numberwithin{table}{section} % Number tables within sections (i.e. 1.1, 1.2, 2.1, 2.2 instead of 1, 2, 3, 4)

%\setlength\parindent{0pt} % Removes all indentation from paragraphs - comment this line for an assignment with lots of text


\setlength\parskip{1pt}

%----------------------------------------------------------------------------------------
%	TITLE SECTION
%----------------------------------------------------------------------------------------

\newcommand{\SILMNum}{\nth{5}}
\newcommand{\SILMYear}{2023}
\newcommand{\SILMDate}{Monday, the \nth{3} of July 2023}

\usepackage{authblk}
\newcommand{\horrule}[1]{\rule{\linewidth}{#1}} % Create horizontal rule command with 1 argument of heig
\title{\vspace{-0.5cm}SILM Workshop on Security of Software/Hardware Interfaces}
\date{}
% \author{Guillaume Hiet, Pierre Wilke, Frédéric Besson, Benoît Montagu, Pascal Cotret} % Your name

\author{Guillaume Hiet}
\affil{CIDRE team, CentraleSupélec/Inria, IRISA, France}

\author{Jan Tobias M\"uhlberg}
\affil{imec-DistriNet, KU Leuven, Belgium}



\newcommand\etal{\emph{et al.}\xspace}
\newcommand\eg{\emph{e.g.}\xspace}
\newcommand\ie{\emph{i.e.}\xspace}

\begin{document}
\maketitle % Print the title



%\begin{abstract}

%\end{abstract}
\begin{center}
 \textbf{Contacts}:  \url{guillaume.hiet@centralesupelec.fr},
\url{jantobias.muehlberg@cs.kuleuven.be}
\end{center}

\noindent This is a proposal for a \emph{full-day} EuroS\&P \SILMYear{} workshop,
preferably to be held before the main conference, on \emph{\SILMDate}.

\section{Aim, scope and context}
%    Workshop title
%    Aim and scope
\subsection{Context and past events}

The security of software and hardware components used to be considered as
different problems and have been studied by distinct scientific
communities. However, it is increasingly important to combine both software
and hardware aspects of computer science to deal with new software attacks
that can be launched remotely and do not require any physical access to the
device. This emphasizes the need to study both the attack and the defense
aspects of the security of software/hardware interfaces.

This poses several challenges. First, hardware platforms tend to be more
and more complex and the behavior specifications of the different hardware
components are not always publicly available. Such specifications can also
be incomplete, incorrect, or imprecise. Thus, different approaches have to
be proposed by researchers to recover the expected behavior or the state of
hardware components.

Another important challenge is to assess the security level of these
hardware platforms against software attacks. The study of existing and new
types of vulnerabilities in hardware platforms is crucial. Analyzing 
software attacks that can use such vulnerabilities is also important to
evaluate their real-world feasibility and impact.

As a third pilar, the study of defense mechanisms is also fundamental to
building secure systems. Modern platforms tend to embed more and more
security mechanisms in hardware components. This raises interesting
questions such as: What type of security mechanisms could be implemented in
hardware platforms at an acceptable cost? What are precisely the security
properties guaranteed by such mechanisms? Software countermeasures can also
be used in systems and applications executed on the hardware platform to
protect them against software attacks exploiting hardware vulnerabilities.

Guillaume Hiet, the first author of this proposal organized a thematic
research semester on these topics\footnote{\url{https://silm.inria.fr/}},
called SILM hereafter. This semester is funded by the
DGA\footnote{\url{https://www.defense.gouv.fr/dga}}, the Government Defence
procurement and technology agency. It is  operated by
Inria\footnote{\url{https://www.inria.fr/en/}}, the French national
research institute for the digital sciences, on behalf of the partners of
the PEC (\textit{Pôle d'Excellence
Cyber})\footnote{\url{https://www.pole-excellence-cyber.org/presentation-du-pole/pole-dexcellence-cyber-at-a-glance/}}
General Partnership Agreement. The PEC is a French cluster of governmental,
industrial and academic partners on cyber-security.

The SILM thematic semester is dedicated to the security of
software/hardware interfaces and we focus more particularly on the three
following axes:

\begin{enumerate}
%
    \item Analyzing the behavior and the state of hardware components
using, e.g. trace mechanisms, fuzzing, reverse-engineering techniques, or
side channel analyses;
%
    \item Studying the hardware vulnerabilities and the software attacks
that can exploit them: e.g. side-channels, fault injections, or
exploitation of unspecified behavior;
%
    \item Detecting and preventing software attacks using dedicated
hardware components. Proposing software countermeasures to protect from
hardware vulnerabilities.
%
\end{enumerate}

In the context of this semester, we have already organized different events
on these topics:
%
\begin{itemize}
%
    \item A SILM Summer School (July 8-12, 2019 ):
\url{https://silm-school.inria.fr/}
%
    \item A \nth{1} edition of the SILM Workshop (November, 2019):
\url{https://silm-workshop.inria.fr/}
%
    \item A \nth{2} edition of the SILM Workshop (September 2020, with
EuroS\&P): \\
\url{https://silm-workshop-2020.inria.fr/}
%
    \item The \nth{3} and \nth{4} edition of the SILM Workshop (2021 and 2022, both with
EuroS\&P, together
with Jan Tobias M\"uhlberg): \\
\url{https://silm-workshop-2021.inria.fr/};
\url{https://silm-workshop.github.io/}
%
    \item A regular seminar: \url{https://silm.inria.fr/silm-seminar/}
%
\end{itemize}

The first edition of the SILM Workshop was organized during a more general
event, the European Cyber Week\footnote{\url{https://www.european-cyber-week.eu/}}.
The \nth{2}, \nth{3} and \nth{4} edition were co-localized with IEEE Euro S\&P as of 2020. 
% 
The summer school, the seminar and the first edition of the SILM workshop
were organized in Rennes, France. Attending those events was free of
charge. The \nth{2} and \nth{3} edition of the SILM workshop were organized as
online-events with IEEE Euro S\&P. Finally, the \nth{4} edition took place as an in-person event again, with some ad-hoc hybrid component.

The programs of the events are available on their respective websites. We
also published the slides and the video recordings of the presentations on
our websites, if the speakers allowed us to do that. All the presentations
have been given by invited speakers from academia, governmental agencies
and industry. The number of participants was the following:
%
\begin{itemize}
%
    \item SILM School : 45 to 70 participants depending on the sessions (45
participants were registered for the whole event, including the labs),
including 13 invited speakers;
%
    \item \nth{1} edition of the SILM Workshop: 76 participants, including
12 invited speakers.
%
    \item \nth{2} edition of the SILM Workshop: 17 participants, including
2 invited speakers.
%
    \item \nth{3} edition of the SILM Workshop: $\approx$ 25 participants,
incl. 3 invited speakers and 2 extra panelists.
%
    \item \nth{4} edition of the SILM Workshop: $\approx$ 25 participants.
\todo{check!}
%
\end{itemize}

We would like to organize a \SILMNum{} edition of the SILM workshop, again
co-located with EuroS\&P, soliciting short papers and full papers for novel and unpublished with a focus on ongoing research with a potential to spark discussions and new collaborations. SILM will also be an
opportunity for young researchers to present their ongoing work.

\subsection{Aim and scope of the proposed workshop}

The purpose of the SILM workshop is to share experience, tools and
methodology to handle security in software/hardware interfaces. On one
hand, we need to better assess the security guarantees provided by existing
hardware architectures against software attacks, especially attacks against
micro-architecture. This can be achieved by identifying new vulnerabilities
using reverse engineering, fuzzing or other attack approaches. On the other
hand, we also need to propose new architectures offering a better
resilience against software attacks. These architectures should rely on
hardware-based security mechanisms to protect the software stack. One of
the challenges is to formally specify and verify the security guarantees
offered by such architectures.

The goal of this \SILMNum{} edition of the SILM workshop is to provide a forum
for  researchers and practitioners from academia, industry and government
that work on the security of software/hardware interfaces.

According to the audience of the previous workshops and events organized
during the SILM semester, we expect 20 to 40 participants, in particular if
we would be able to hold a physical meeting in Genoa.

For the last edition of the SILM workshop, we received 7 submissions and we
accepted one full paper (10 pages) and four short paper (6 pages). We also
had two invited speakers.
\todo{check!}

%    Past history of the workshop, if any, including the number of submissions, number of accepted papers, and number of registered attendees for the most recent edition of the workshop, and last year’s website
% 

\section{Format}

We would like to organize a one day event mixing presentation from invited
speakers, regular research papers and short
papers/work-in-progress/industry experience reports. The precise format
will depend on the number of submissions but we target the following
format:
%
\begin{itemize}
    \item 3 invited speakers (45 minutes)
    \item 2 full research papers (25 minutes)
    \item 4 short presentations (15 minutes)
    \item 1 panel debate or discussion round (50 minutes)
\end{itemize}

Depending on the submissions, we could accept more research papers and
fewer short presentations. Expenses not covered by the support of Euro S\&P
will be covered by the budget of the SILM thematic semester.

The proposed agenda is the following one:
%
\begin{itemize}
    \item 08:30 - 08:45 : Registration
    \item 08:45 - 09:00 : Opening remarks
    \item 09:00 - 10:00 : Keynote
    \item 10:00 - 10:30 : Coffee break
    \item 10:30 - 12:30 : Presentations
    \item 12:30 - 14:00 : Lunch
    \item 14:00 - 15:00 : Keynote
    \item 15:00 - 16:00 : Panel debate / Discussion
    \item 16:00 - 16:30 : Coffee break
    \item 16:30 - 17:30 : Keynote
    \item 17:30 - 18:00 : Closing remarks
\end{itemize}
%
Of course, we will adapt this agenda depending on the local organization constraints.

\section{Reviewing and publication}

We would like to propose two categories of submissions:
%
\begin{itemize}
%
    \item Regular papers describing fully developed work and complete
results (10 pages, references included, IEEE format);
%
    \item Short papers, position papers, industry experience reports,
work-in-progress submissions (6 pages, references included, IEEE format)
%
\end{itemize}

All papers should be in English and describe original work that has not
been previously published or submitted elsewhere. The submission category
should be clearly indicated. All submissions will be fully reviewed by
members of the Program Committee.

We expect between 10 and 20 submissions for SILM \SILMYear{}. Each submission will
be reviewed by three different members of the PC.
Papers will appear in IEEE Xplore in a companion volume to the regular Euro
S\&P proceedings.

%    Expected number of submissions, paper length (in IEEE conference proceedings format), and number of attendees
%    Plan for the review process and the publication of proceedings

\section{Organization}
%    Program chair(s), and if available, tentative program and organization committees
%    Biographical sketch for each main organizer, describing relevant qualifications in research and conference/workshop organizing experience

\subsection{Organization Committee}

The Organization Committee will be chaired by Guillaume Hiet and Jan Tobias
M\"uhlberg.

\emph{Guillaume Hiet}\footnote{\url{http://guillaume.hiet.fr/}} is an
Assistant Professor at CentraleSupélec and a member of the IRISA/Inria
CIDRE research team in Rennes Brittany, France. Prior to that he was a
computer security expert at AMOSSYS SAS, a French security consulting
company, from 2008 to 2010. He holds a PhD from Rennes 1 University, a
Diplôme d'Ingénieur (MEng) and a Master Recherche (MSc) from Supélec as
well as a Diplôme d'Ingénieur from ENSAM. He teaches cyber-security at
CentraleSupélec and in other institutions. His research interests are in
network and computer security, with a focus on security monitoring. He is
particularly interested in the security of low-level components such as OS,
firmware and hardware. He was involved in the PC of RAID 2011 and RAID 2012
security conference. He is the chair holder of the SILM thematic semester.

\emph{Jan Tobias
M\"uhlberg}\footnote{\url{https://distrinet.cs.kuleuven.be/people/JanTobiasMuhlberg}}
is a Research Manager for Embedded Systems Security at imec-DistriNet at
the Department of Computer Science of KU Leuven (BE). He is active in the
fields of embedded systems security, hardware/software co-design for
security, verification and validation of software systems. Tobias is
particularly interested in security architectures for safety-critical
embedded systems and for the Internet of Things, and in the concept of
sustainability in the information and communications technology,
specifically in the context of security and privacy. Before joining
DistriNet he worked as a researcher at the University of Bamberg (DE),
obtained his Ph.D. from the University of York (UK) and worked as a
researcher at the University of Applied Sciences in Brandenburg (DE), where
he also obtained his Masters degree. Tobias has been a co-chair of SILM
2021, is a co-organiser of the SICT Summer School (since 2020), the
QA\&Test conferences (since 2018), and several seminar series at KU Leuven.

\subsection{Program Committee}

We are working on the composition of the Program Committee. We currently
strive to increase the geographic distribution and diversity of this
workshop's PC and the invited speakers. As of now we have contacted the
following distinguished researchers in the field, most of whom have
accepted to be part of the Program Committee of this workshop:

%Invite:
%Alasdair Armstrong, Cambridge
%David Goltzsche (Braunschweig) -> https://www.doc.ic.ac.uk/~vsartako/
%someone from the Robert Watson/Simon Moore group (Cambridge)
%Shweta Shinde or Srdjan Capkun (ETH)

\begin{itemize}
    \item Guillaume Bouffard, ANSSI
    \item Pascal Cotret, ENSTA Bretagne
    \item Damien Couroussé, CEA
    \item Chris Dalton, HP Labs
    \item{Lesly-Ann Daniel, KU Leuven}
    \item Lucas Davi, University of Duisburg-Essen
    \item Karine Heydemann, LIP6
    \item Guillaume Hiet, CentraleSupélec/Inria (Co-Chair)
    \item Vianney Lapôtre, Univ. South Brittany
    \item Jan Tobias Mühlberg, KU Leuven (Co-Chair)
    \item Cristofaro Mune, Raelize B.V.
    \item Simon Rokicki, ENS Rennes
    \item Volker Stolz, HVL
    \item{Marcus V\"olp, Uni Luxembourgh}
    \item Pierre Wilke, CentraleSupélec/Inria
    \item Yuval Yarom, University of Adelaide and Data61
%    \item Guy Gogniat, Univ. South Brittany
%    \item Jean-Louis Lanet, Inria
%    \item Yves-Alexis Perez, ANSSI
%    \item Kaveh Razavi,  ETH Zürich 
%    \item Jan Reineke, Saarland University
%    \item Erven Rohou, Inria
%    \item André Seznec, Inria
%    \item Arnaud Tisserand, CNRS
%    \item Frédéric Tronel, CentraleSupélec/Inria
%    \item Yossi Oren, Ben-Gurion University
\end{itemize}


\section{Draft CFP}
%   Draft CFP articulating the scope and topics covered by the workshop

\verbatiminput{cfp.txt}


\end{document}


