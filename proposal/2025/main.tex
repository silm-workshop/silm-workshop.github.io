% License:
% CC BY-NC-SA 3.0 (http://creativecommons.org/licenses/by-nc-sa/3.0/)
%
%%%%%%%%%%%%%%%%%%%%%%%%%%%%%%%%%%%%%%%%%

\documentclass[a4paper,11pt]{article} 
\usepackage[margin=2cm]{geometry}
\usepackage[T1]{fontenc} 
\usepackage[utf8]{inputenc}
\usepackage{fourier} 
\usepackage[english]{babel} 
\usepackage[english=american]{csquotes}
\usepackage{amsmath,amsfonts,amsthm}
\usepackage{paralist}
\usepackage{caption}
\usepackage{subcaption}
\usepackage{graphicx}
\usepackage{float}
\usepackage{blindtext} 
\usepackage{tabularx}
\usepackage{tabto}
\usepackage{boldline}

\usepackage{verbatim}
\usepackage{etoolbox}
\usepackage{todonotes}
\usepackage{nth}

\PassOptionsToPackage{hyphens}{url}
\usepackage[]{hyperref} 

\usepackage{fancyhdr} 
\pagestyle{fancyplain}
\fancyhead{} 
\fancyfoot[L]{} 
\fancyfoot[C]{}
\fancyfoot[R]{\thepage} 
\renewcommand{\headrulewidth}{0pt}
\renewcommand{\footrulewidth}{0pt} 
\setlength{\headheight}{13.6pt} 

\setlength\parskip{1pt}

\usepackage{authblk}
\newcommand{\horrule}[1]{\rule{\linewidth}{#1}} 

\newcommand\etal{\emph{et al.}\xspace}
\newcommand\eg{\emph{e.g.}\xspace}
\newcommand\ie{\emph{i.e.}\xspace}



% ---------------------------------------------------------------------------
\newcommand{\SILMNum}{\nth{6}}
\newcommand{\SILMYear}{2025}
\newcommand{\SILMDate}{Thursday, the \nth{25} of September 2025}

\title{H3S Workshop on Hardware-Supported Software Security}

\date{\today}

\author{Guillaume Hiet}
\affil{CentraleSupélec/Inria, Rennes, France \\
\url{guillaume.hiet@centralesupelec.fr}}

\author{Yuko Hara}
\affil{Institute of Science, Tokyo, Japan\\
\url{hara@cad.ict.e.titech.ac.jp}}

\author{Jan Tobias M\"uhlberg}
\affil{Universit\'e Libre de Bruxelles, Brussels, Belgium \\
\url{jan.tobias.muehlberg@ulb.be}}
% ---------------------------------------------------------------------------



% ---------------------------------------------------------------------------
% ---------------------------------------------------------------------------
\begin{document}

\maketitle

% ---------------------------------------------------------------------------
\begin{abstract}
%
The H3S Worskhop on Hardware-Supported Software Security will be the successor of the 
SILM Workshop on Security of Software/Hardware Interfaces (originally
in French: \enquote{S\'ecurit\'e des Interfaces Logiciel/Mat\'eriel}). It 
solicits submissions -- full papers and work-in-progress submissions --
around attacks and defences that involve the interaction of
hardware and software, such as defence mechanisms that
involve the co-design of hardware and software or software attacks that exploit hardware vulnerabilities. We have already organised
five iterations of the SILM workshop, mostly co-located with IEEE EuroS\&P.
SILM usually attracts around 10 paper submissions and 30+ participants.
Previous editions of SILM have been rather informal but inspiring events
with a focus on fostering interaction and collaboration, and sharing
knowledge and discussing new directions in our field. As organisers, we have
been encouraging collaborations with industry. We seek to create an
inclusive environment and adapt the scope of H3S to include new
challenges and approaches to developing secure systems.

This is a proposal for a \emph{full-day} ESORICS \SILMYear{}
workshop, preferably to be held on
\emph{\SILMDate}.
%
\end{abstract}

% ---------------------------------------------------------------------------
% ---------------------------------------------------------------------------
\section{Aim, Scope and Context}
%
% ---------------------------------------------------------------------------
\subsection{Context and Past Events}
%
The security of software and hardware components used to be considered as
different problems and have been studied by distinct scientific
communities. However, it is increasingly important to combine both software
and hardware aspects of computer science to deal with new software attacks
that can be launched remotely and do not require any physical access to the
device. This emphasises the need to study both the attack and the defence
aspects of the security of software/hardware interfaces.

This poses several challenges. First, hardware platforms tend to be more
and more complex, and the behaviour specifications of the different hardware
components are not always publicly available. Such specifications can also
be incomplete, incorrect, or imprecise. Thus, different approaches have to
be proposed by researchers to recover the expected behaviour or the state of
hardware components.

Another important challenge is to assess the security level of these
hardware platforms against software attacks. The study of existing and new
types of vulnerabilities in hardware platforms is crucial. Analysing
software attacks that can use such vulnerabilities is also important to
evaluate their real-world feasibility and impact.

As a third pillar, the study of defence mechanisms is also fundamental to
building secure systems. Modern platforms tend to embed more and more
security mechanisms in hardware components. This raises interesting
questions such as: What type of security mechanisms could be implemented in
hardware platforms at an acceptable cost? What are precisely the security
properties guaranteed by such mechanisms? Software countermeasures can also
be used in systems and applications executed on the hardware platform to
protect them against software attacks exploiting hardware vulnerabilities.

Guillaume Hiet and Jan Tobias M\"uhlberg are co-organising the SILM
workshop since 2021.
Guillaume Hiet organised a thematic
research semester on these topics\footnote{\url{https://silm.inria.fr/}},
called SILM hereafter. This semester was funded by the
DGA\footnote{\url{https://www.defense.gouv.fr/dga}}, the Government Defence
procurement and technology agency. It was  operated by
Inria\footnote{\url{https://www.inria.fr/en/}}, the French national
research institute for the digital sciences, on behalf of the partners of
the PEC (\textit{Pôle d'Excellence
Cyber})\footnote{\url{https://www.pole-excellence-cyber.org/presentation-du-pole/pole-dexcellence-cyber-at-a-glance/}}
General Partnership Agreement. The PEC is a French cluster of governmental,
industrial and academic partners on cyber-security.

The SILM thematic semester was dedicated to the security of
software/hardware interfaces and more particularly on the three
following axes:

\begin{enumerate}
%
    \item Analysing the behaviour and the state of hardware components
using, e.g. trace mechanisms, fuzzing, reverse-engineering techniques, or
side channel analyses;
%
    \item Studying the hardware vulnerabilities and the software attacks
that can exploit them: e.g. side-channels, fault injections, or
exploitation of unspecified behaviour;
%
    \item Detecting and preventing software attacks using dedicated
hardware components. Proposing software countermeasures to protect from
hardware vulnerabilities.
%
\end{enumerate}

In the context of this semester, we have already organised different events
on these topics:
%
\begin{itemize}
%
    \item A SILM Summer School (July 8-12, 2019 ):
\url{https://silm-school.inria.fr/}
%
    \item A \nth{1} edition of the SILM Workshop (November, 2019):
\url{https://silm-workshop.inria.fr/}
%
    \item A \nth{2} edition of the SILM Workshop (September 2020, with
EuroS\&P): \\ \url{https://silm-workshop-2020.inria.fr/}
%
    \item{The \nth{3}, \nth{4}, and \nth{5} edition of the SILM Workshop (2021 and
2022, 2023, all with EuroS\&P, together with Jan Tobias M\"uhlberg): \\
\url{https://silm-workshop-2021.inria.fr/};
\url{https://silm-workshop.github.io/}
  }
%
    \item A regular seminar: \url{https://silm-seminar.gitlabpages.inria.fr/}
%
\end{itemize}

The first edition of the SILM Workshop was organised during a more general
event, the European Cyber
Week\footnote{\url{https://www.european-cyber-week.eu/}}.  The \nth{2},
\nth{3} and \nth{4} edition were co-located with IEEE Euro S\&P as of 2020. 
% 
The summer school, the seminar and the first edition of the SILM workshop
were organised in Rennes, France. Attending those events was free of
charge. The \nth{2} and \nth{3} edition of the SILM workshop were organised
as online-events with IEEE Euro S\&P. Finally, the \nth{4} and \nth{5}
editions took
place as an in-person event again, with an ad-hoc hybrid component.

The programs of the events are available on their respective websites. We
also published the slides and (partial) video recordings of the presentations on
our websites, if the speakers allowed us to do that. The presentations
have been given predominantly by academic researchers who submitted
double-blinded manuscripts fro review by the PC, and by invited speakers from academia, governmental agencies
and industry. The number of participants was the following:
%
\begin{itemize}
%
    \item SILM School: 45 to 70 participants depending on the sessions (45
participants were registered for the whole event, including the labs),
including 13 invited speakers;
%
    \item \nth{1} edition of the SILM Workshop: 76 participants, incl.
12 invited speakers.
%
    \item \nth{2} edition of the SILM Workshop: 17 participants, incl.
2 invited speakers.
%
    \item \nth{3} edition of the SILM Workshop: $\approx$ 25 participants,
incl. 3 invited speakers and 2 extra panellists.
%
    \item \nth{4} edition of the SILM Workshop: $\approx$ 30 participants,
incl. 4 invited speakers.
%
    \item \nth{5} edition of the SILM Workshop: $\approx$ 30 participants,
incl. 2 invited speakers.
\end{itemize}

We want to organise a new workshop, HS3, this time co-located with ESORICS. Several reasons motivate our desire to change the name and co-location of the workshop. First, the historical name, SILM, is based on a French acronym, and according to feedback from our community, the connection between the workshop name and its acronym was not obvious. Second, the initial scope was quite broad and overlapped with topics already covered by other workshops or conferences. The new name emphasises hardware support for software security (though the workshop’s scope remains large, covering most topics on security at the hardware/software interface).

Moreover, we had to cancel the 2024 edition of SILM, which we had planned to co-locate with IEEE Euro S\&P, because new workshops co-located with this conference covered similar topics. Additionally, from an organisational perspective, we prefer to hold an event in September rather than in mid-July.
HS3 will solicit short papers and full papers for novel and unpublished with a focus on ongoing research with the potential to spark
discussions and new collaborations. HS3 will also allow
young researchers to present their ongoing work.

% ---------------------------------------------------------------------------
\subsection{Aim and Scope of the Proposed Workshop}
%
The purpose of the HS3 workshop is to share experience, tools and
methodology to handle security in software/hardware interfaces. On one
hand, we need to better assess the security guarantees provided by existing
hardware architectures against software attacks, especially attacks against
micro-architecture. This can be achieved by identifying new vulnerabilities
using reverse engineering, fuzzing or other attack approaches. On the other
hand, we also need to propose new architectures offering a better
resilience against software attacks. These architectures should rely on
hardware-based security mechanisms to protect the software stack. One of
the challenges is to formally specify and verify the security guarantees
offered by such architectures.

The goal of this HS3 workshop is to provide a forum
for  researchers and practitioners from academia, industry and government
that work on the security of software/hardware interfaces.

% \paragraph{Special Theme \SILMYear{}: SILM for Telecommunications.}
% %
% As a special theme for SILM \SILMYear{}, we solicit submissions that
% present novel research that develops or relies on hardware/software
% interfaces to improve security and dependability in the telecommunications
% sector. In particular, work that aims at 5G, edge and cloud scenarios where
% strict requirements on bandwidth, latency, and availability need to be met,
% are of special interest for SILM this year. Submissions must be within the
% general scope of SILM \SILMYear{} as stated above.

\paragraph{Expected Number of Participants.}
%
According to the audience of the previous workshops and events organised
during the SILM semester, we expect 20 to 40 participants.

\paragraph{Expected Number of Submissions.}
%
We expect around 10 submissions; the majority of these submissions will be
short papers.
%
For the last two editions of the SILM workshop: 
In 2023 we received 11 submissions
and we accepted two full paper (10 pages) and six short paper (6 pages).
In 2022 we received 7 submissions
and we accepted one full paper (10 pages) and four short paper (6 pages).
We usually complement the programme with two to
four invited speakers or a panel discussion.

\paragraph{Justification for a Full-Day Workshop}
%
The last two editions of SILM have been very lively events that involved
interventions beyond paper presentations and invited talks, such moderated
and unmoderated discussion rounds and ad-hoc organised short presentations
by the participants. Some of these interactions resulted in future
collaborations, in particular for junior researchers. We see it as an
important responsibility of the SILM workshop to further foster these
interactions and to provide the time and space for a full-day experience
that also involves joint lunch and dinner.


% ---------------------------------------------------------------------------
% ---------------------------------------------------------------------------
\section{Format}
%
We would like to organise a one day event mixing presentation from invited
speakers, regular research papers and short
papers/work-in-progress/industry experience reports. The precise format
will depend on the number of submissions but we target the following
format:
%
\begin{itemize}
    \item 3 invited speakers (45 minutes)
    \item 2 full research papers (25 minutes)
    \item 4 short presentations (15 minutes)
    \item 1 panel debate or discussion round (50 minutes)
\end{itemize}

We could accept more research papers and
fewer short presentations depending on the submissions. Expenses not covered by the support of ESORICS
will be covered by the budget of the SUSHI Inria team or ULB.

The proposed agenda is the following one:
%
\begin{itemize}
    \item 08:30 - 08:45 : Registration
    \item 08:45 - 09:00 : Opening remarks
    \item 09:00 - 10:00 : Keynote
    \item 10:00 - 10:30 : Coffee break
    \item 10:30 - 12:30 : Presentations
    \item 12:30 - 14:00 : Lunch
    \item 14:00 - 15:00 : Keynote
    \item 15:00 - 16:00 : Panel debate / Discussion / Short talks
    \item 16:00 - 16:30 : Coffee break
    \item 16:30 - 17:30 : Keynote
    \item 17:30 - 18:00 : Closing remarks
\end{itemize}
%
Of course, we will adapt this agenda depending on the local organisation constraints.


% ---------------------------------------------------------------------------
% ---------------------------------------------------------------------------
\section{Reviewing \& Publication}
%
We would like to propose two categories of submissions:
%
\begin{itemize}
%
    \item Regular papers describing fully developed work and complete
results (20 pages, references included, LNCS format);
%
    \item Short papers, position papers, industry experience reports,
work-in-progress submissions (10 pages, references included, LNCS format)
%
\end{itemize}

All papers should be in English and describe original work that has not
been previously published or submitted elsewhere. The submission category
should be clearly indicated. Members of the Program Committee will fully review all submissions.

We expect between 10 and 20 submissions for HS3. Three different members of the PC will review each submission.  If
possible, papers will appear in LNCS.

In the past years, we offered authors the option to present their papers but
not have them published in the official Euro S\&P proceedings, so as to
encourage authors to present and discuss ongoing work and leave them the
opportunity to publish their research once it is completed. We will continue to follow
this policy for HS3.


% ---------------------------------------------------------------------------
% ---------------------------------------------------------------------------
\section{Organisation}
%
% ---------------------------------------------------------------------------
\subsection{Organisation Committee}
%
The Organisation Committee will be chaired by Guillaume Hiet, Yuko Hara, and Jan Tobias
M\"uhlberg. Merve G\"ulmez will support the committee as Publicity Chair.

\textbf{Guillaume Hiet\footnote{\url{http://guillaume.hiet.fr/}}} works as a
Professor at CentraleSupélec. He is the team leader of the IRISA/Inria SUSHI research
team in Rennes, Brittany, France. Before that, he was a computer security
expert at AMOSSYS SAS, a French security consulting company, from 2008 to
2010. He holds an HDR and a PhD from Rennes 1 University, a Diplôme d'Ingénieur (MEng)
and a Master Recherche (MSc) from Supélec as well as a Diplôme d'Ingénieur
from ENSAM. He teaches cyber-security at CentraleSupélec and in other
institutions. His research interests are in network and computer security,
focusing on security monitoring. He is particularly interested in the
security of low-level components such as OS, firmware and hardware. He was
involved in the PC of RAID 2011 and RAID 2012 security conference. He was
the chairholder of the SILM thematic semester.

\textbf{Yuko Hara\footnote{\url{https://sites.google.com/view/yukohara/}}}  works as an Associate Professor in the Department of Information and Communications Engineering, School of Engineering, at the Institute of Science Tokyo, Japan.

\textbf{Jan Tobias
M\"uhlberg\footnote{\url{https://cybersecurity.ulb.ac.be/jan-tobias-muhlberg/}}}
works as a Professor for Embedded Systems Security at Universit\'e Libre de
Bruxelles (BE) and as Research Manager for Embedded Systems Security at
imec-DistriNet at the Department of Computer Science of KU Leuven (BE). He
is active in the fields of embedded systems security, hardware/software
co-design for security, verification and validation of software systems.
Tobias is particularly interested in security architectures for
safety-critical embedded systems and for the Internet of Things, and in the
concept of sustainability in information and communications technology,
specifically in the context of security and privacy. Before joining
DistriNet, he worked as a researcher at the University of Bamberg (DE),
obtained his Ph.D. from the University of York (UK) and worked as a
researcher at the University of Applied Sciences in Brandenburg (DE), where
he also obtained his Masters degree. Tobias has been a co-chair of SILM
since 2021, is a PC member of Euro S\&P 2023, is a co-organiser of the SICT
Summer School (since 2020), the QA\&Test conferences (since 2018), and
several seminar series at KU Leuven.

\textbf{Merve
G\"ulmez\footnote{\url{https://gulmezmerve.github.io/}}}
is a dependability researcher at Ericsson Research (SE) and KU Leuven (BE).
Merve is currently doing her PhD on low-level security and availability
mechanisms for cloud- and 5G telecommunications infrastructures.


% ---------------------------------------------------------------------------
\subsection{Draft Program Committee}
%
We are working on the composition of the Program Committee. We 
strive to increase the geographic distribution and diversity of this
workshop's PC and the invited speakers. We plan to contact the
following distinguished researchers in the field who were part of the previous SILM PC:

% Invite:
% Alasdair Armstrong, Cambridge
% David Goltzsche (Braunschweig) -> https://www.doc.ic.ac.uk/~vsartako/
% someone from the Robert Watson/Simon Moore group (Cambridge)
% Shweta Shinde or Srdjan Capkun (ETH)
% Cl\'ementine Maurice

\begin{itemize}
    \item Pascal Cotret, ENSTA Bretagne
    \item Chris Dalton, HP Labs
    \item{Lesly-Ann Daniel, KU Leuven}
    \item{Merve G\"ulmez, Ericsson Research/KU Leuven (Publicity)}
    \item Karine Heydemann, LIP6
    \item Guillaume Hiet, CentraleSupélec/Inria (Co-Chair)
    \item Vianney Lap\^otre, Univ. South Brittany
    \item{Cl\'ementine Maurice, Inria}
    \item Jan Tobias M\"uhlberg, KU Leuven (Co-Chair)
    \item Cristofaro Mune, Raelize B.V.
    \item Kaveh Razavi,  ETH Z\"urich 
    \item Simon Rokicki, ENS Rennes
    \item Volker Stolz, HVL
    \item{Marcus V\"olp, Uni Luxembourgh}
    \item Pierre Wilke, CentraleSup\'elec/Inria
    \item Yuval Yarom, University of Adelaide and Data61
%    \item Guillaume Bouffard, ANSSI
%    \item Guy Gogniat, Univ. South Brittany
%    \item Damien Courouss\'e, CEA
%    \item Lucas Davi, University of Duisburg-Essen
%    \item Jean-Louis Lanet, Inria
%    \item Yves-Alexis Perez, ANSSI
%    \item Jan Reineke, Saarland University
%    \item Erven Rohou, Inria
%    \item Andr\'e Seznec, Inria
%    \item Arnaud Tisserand, CNRS
%    \item Fr\'ed\'eric Tronel, CentraleSup\'elec/Inria
%    \item Yossi Oren, Ben-Gurion University
\end{itemize}


% ---------------------------------------------------------------------------
% ---------------------------------------------------------------------------
\section{Draft CFP}
%
Note that the paper submission deadline is chosen to allow for a two-weeks
extension within the deadlines set out in the call for workshop proposals.
%
\verbatiminput{cfp.txt}


\end{document}
% ---------------------------------------------------------------------------
% ---------------------------------------------------------------------------
% ---------------------------------------------------------------------------

